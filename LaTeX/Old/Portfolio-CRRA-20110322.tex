\documentclass{bejournal}
\usepackage{subfigure}
% CDCPrivate: Derivations and verification are available in ./Software/Mathematica/Derivations.nb

\input CDCDocStartForBE.tex 

\input HandoutStart.tex
\input DocStartExam.tex

\begin{document}
\input HandoutHeader.tex

\begin{verbatimwrite}{\jobname.title}
Portfolio Choice With CRRA Utility (Merton-Samuelson)
\end{verbatimwrite}
\medskip\medskip\medskip

\input HandoutName.tex

\begin{verbatimwrite}{./body.tex}

  \cite{merton:restat} and \cite{samuelson:portfolio} study the
  optimal portfolio choice of a consumer with Constant Relative Risk
  Aversion utility $\util(c) = (1-\CRRA)^{-1}c^{1-\CRRA}$ with
  relative risk aversion $\CRRA>1$.  This consumer has assets at the
  end of period $t$ equal to $a_{t}$ and is deciding how much to
  invest in a risky asset with a lognormally distributed return factor
  $\Risky_{t+1}$, $\log \Risky_{t+1} = \risky_{t+1} \sim
  \mathcal{N}(\risky-\Evarr/2,\sigma^{2}_{\risky})$,\footnote{\MathFacts~\handoutM{ELogNorm}
    tells us that a variable with this lognormal distribution has an
    expectation $\Ex_{t}[e^{\risky_{t+1}}]=e^{\risky}=\Risky$ (where
    variables like $\Risky$ without a subscript are the time-invariant
    mean).} versus a riskfree asset that earns return factor
  $\Rfree=e^{\rfree}$.  Importantly, the consumer is assumed to have
  no labor income and to face no risk except from the investment in
  the risky asset.\footnote{In addition to labor income risk, this
    rules out, for example, health expense risk which recent research
    has argued is important later in life.}$^{,}$\footnote{Riskless
    labor income can trivially be added to the problem, because its
    risklessness means that (in the absence of liquidity constraints)
    it is indistinguishable from a lump sum of extra assets with a
    value equal to the present discounted value of the riskless labor
    income.}

Both papers consider a multiperiod optimization problem (Merton is in
continuous time, Samuelson discrete), but here we examine the problem
of a consumer in the second-to-last period of life (the insights carry
over to the multiperiod case).  

If the consumer invests proportion $\riskyshare$ in the risky asset, 
spending all available resources in the last period of life $t+1$ will yield:
\begin{eqnarray*}
        c_{t+1} & = & \left(\Rfree(1-\riskyshare)+\Risky_{t+1}\riskyshare\right)a_{t}
\\ & = & \underbrace{\left(\Rfree+(\Risky_{t+1}-\Rfree)\riskyshare\right)}_{\equiv \Rport_{t+1}}a_{t}
\end{eqnarray*}
where $\Rport_{t+1}$ is the realized return factor for the portfolio.
Finally defining the logarithmic excess return as $\eprem_{t+1}\equiv
\risky_{t+1}-\rfree$, \cite{cvAppendix} show that the log of the
portfolio return factor can be well approximated by
\begin{eqnarray}
 \rport_{t+1} \equiv \log \Rport_{t+1} & \approx & \rfree + \eprem_{t+1} \riskyshare + \riskyshare(1-\riskyshare)\sigma^{2}_{\risky}/2 \label{eq:CampVic}
\end{eqnarray}
which is greater than $\rfree + \eprem_{t+1} \riskyshare$ because it is the log of the weighted average of $\Rfree$ and $\Risky$ and since the logarithm is a concave function, $\log \Rport < \riskyshare \log \Risky + (1-\riskyshare) \log \Rfree$ by Jensen's inequality.

Under these assumptions, the expectation as of time $t$ of utility in period $t+1$
is:
\begin{eqnarray}
  \Ex_{t}[\util(c_{t+1})] & \approx & (1-\CRRA)^{-1}\Ex_{t}\left[\left(a_{t}e^{\rfree}e^{\riskyshare \eprem_{t+1}+\riskyshare(1-\riskyshare)\sigma^{2}_{\risky}/2 }\right)^{1-\CRRA}\right] \notag
\\                      & \approx & (1-\CRRA)^{-1}\Ex_{t}\left[(a_{t}\Rfree)^{1-\CRRA}\left( e^{\riskyshare \eprem_{t+1}+\riskyshare(1-\riskyshare)\sigma^{2}_{\risky}/2 }\right)^{1-\CRRA}\right] \notag
\\                      & \approx & (1-\CRRA)^{-1}(a_{t}\Rfree)^{1-\CRRA}\Ex_{t}\left[e^{(\riskyshare \eprem_{t+1}+\riskyshare(1-\riskyshare)\sigma^{2}_{\risky}/2)  (1-\CRRA)}\right] \notag
\\                      & \approx & (1-\CRRA)^{-1}(a_{t}\Rfree)^{1-\CRRA}e^{ (1-\CRRA)\riskyshare(1-\riskyshare)\sigma^{2}_{\risky}/2}\Ex_{t}\left[e^{\riskyshare \eprem_{t+1}  (1-\CRRA)}\right]
. \label{eq:exputil}
  \end{eqnarray}

Our foregoing assumptions imply (using \handoutM{SumNormsIsNorm}) that $\eprem_{t+1} \sim \mathcal{N}(\eprem-\Evarr/2,\Evarr)$ (since $\risky_{t+1}$ is normally distributed and $\rfree$ is a
point mass with zero variance),\footnote{The `arithmetic excess return' is the ratio of the realized portfolio return factor $\Rport_{t+1}$ to the riskfree return $\Rfree$.  Because of our careful assumption on the return on the risky asset, the arithmetic excess return does not change when the variance of the risky return changes; but the logarithmic excess return must be adjusted by the $-\sigma^{2}_{\risky}/2$ term in order to preserve the invariance of the arithmetic risky return.} so $\riskyshare \eprem_{t+1}  (1-\CRRA) \sim \mathcal{N}((\eprem - \Evarr/2)\riskyshare  (1-\CRRA),(\riskyshare(1-\CRRA))^{2}\Evarr)$ (see \handoutM{LogELogNormTimes} in \MathFacts).   With a couple of extra lines of derivation we can show that
\begin{eqnarray}
  \Ex_{t}\left[e^{\riskyshare \eprem_{t+1}  (1-\CRRA)}\right] & = & e^{(1-\CRRA)\riskyshare \eprem-(1-\CRRA)\riskyshare\Evarr/2+ ((1-\CRRA)\riskyshare)^{2}\Evarr/2} \notag
\\  & = & e^{(1-\CRRA)\riskyshare \eprem-(1-\CRRA)\riskyshare(1-\riskyshare(1-\CRRA))\Evarr/2} \notag
\\  & = & e^{(1-\CRRA)\riskyshare \eprem-(1-\CRRA)\riskyshare(1-\riskyshare)\Evarr/2-\CRRA (1-\CRRA)\riskyshare^{2}\Evarr/2}. \label{eq:Ex}
\end{eqnarray}

Since for $\CRRA > 1$ the term $(1-\CRRA)^{-1}(a_{t}\Rfree)^{1-\CRRA}$ in
\eqref{eq:exputil} is a negative constant, utility will be maximized
by minimizing the rest of \eqref{eq:exputil} (thus minimizing the size
of the negative number); substituting from \eqref{eq:Ex} for the
expectation in \eqref{eq:exputil} and noting that the product $e^{(1-\CRRA)\riskyshare\Evarr/2}e^{-(1-\CRRA)\riskyshare\Evarr/2}=1$, this means we want to minimize $e^{-(\CRRA-1)\riskyshare \eprem  - (\CRRA-1)(- \CRRA  \riskyshare^{2} \Evarr/2)}$.  
Minimizing this when $\CRRA>1$ is equivalent to maximizing the terms
multiplied by $-(\CRRA-1)$, so our problem reduces to
\[
\max_{\riskyshare}~~ \riskyshare \eprem -\CRRA\riskyshare^{2}\Evarr/2 
\]
with FOC
\begin{eqnarray}
         \eprem-\riskyshare\CRRA\Evarr  & = & 0  \notag \\ 
\riskyshare & = & \left(\frac{\eprem}{\CRRA \Evarr}\right)
. \label{eq:riskyshareMS}
\end{eqnarray}

Equation \eqref{eq:riskyshareMS} says\footnote{This expression differs
  slightly from that derived by \cite{cvAppendix}, because we adjust
  the mean logarithmic return of the risky investment for its variance
  in order to keep its arithmetic mean constant, which makes
  comparisons of alternative levels of risk more transparent.}  that the
consumer allocates more of his portfolio to the high-risk, high-return
asset when
\begin{enumerate}
\item the amount $\eprem$ by which the risky
asset's return exceeds the riskless return is greater
\item the consumer is less risk averse ($\CRRA$ is lower)
\item riskiness $\sigma^{2}_{\risky}$ is less
\end{enumerate}
If there is no excess return, nothing will be put in the risky asset.  Similarly, if 
risk aversion or the variance of the risk is infinity, again nothing 
will be put in the risky asset.\footnote{See the appendix for a figure
  showing the quality of the approximation.}

This formula hints at the existence of an `equity premium puzzle'
(\cite{mehra&prescott:puzzle}).  Interpreting the risky asset as the
aggregate stock market, the annual standard deviation of the log of
U.S.\ stock returns has historically been about $\sigma_{\risky}=0.2$
yielding $\Evarr = 0.04$.  The equity premium over historical periods
has been something like $\eprem = 0.08$ (eight percent).  With risk
aversion of $\CRRA=2$ this formula implies that the share of risky
assets in your portfolio should be $0.08/0.08$ or 100 percent!  The
fact that most people have less than 100 percent of their wealth
invested in stocks is the `stockholding puzzle,' the microeconomic
manifestation of the equity premium puzzle
(\cite{haliassos&bertaut:fewholdstocks}).

A final interesting question is what the expected rate of return on
the consumer's portfolio will be once the portfolio share in risky
assets has been chosen optimally.  Exponentiating the Campbell-Viceira
formula \eqref{eq:CampVic} and dividing by $e^{\rfree}$ to obtain the expected arithmetic `portfolio
excess return' factor we have
\begin{eqnarray}
  e^{\riskyshare(1-\riskyshare)\Evarr/2}\overbrace{\Ex_{t}[e^{\eprem_{t+1}\riskyshare}]}^{e^{\riskyshare \eprem - \riskyshare\Evarr/2+\riskyshare^{2}\Evarr/2}}  & = & e^{\riskyshare(1-\riskyshare)\Evarr/2+\riskyshare \eprem - \riskyshare \Evarr/2+\riskyshare^{2}\Evarr/2} 
\\ & = & e^{\riskyshare \eprem} \label{eq:eport}
\end{eqnarray}
while the variance of the log of the expected excess return factor for the portfolio can similarly be shown to be $\sigma^{2}_{\rport} = \riskyshare^{2} \sigma^{2}_{\risky}.$

Substituting the solution for $\riskyshare$ into \eqref{eq:eport} and taking the log, we have
\begin{eqnarray}
  \riskyshare \eprem & = & \left(\frac{\eprem^{2}}{\CRRA \Evarr}\right)  \notag
\\ & = &  (\eprem/\sigma_{\risky})^{2}/\CRRA \label{eq:epremOpt}
\end{eqnarray}
which is an interesting formula for the excess return of the optimally
chosen portfolio because the object $\eprem/\sigma_{\risky}$ (the
excess return divided by the standard deviation) is a well-known tool
in finance for evaluating the tradeoff between risk and return (the
`Sharpe ratio').  The equation therefore says that the consumer will
choose a portfolio that earns an excess return that is directly
related to the Sharpe ratio and inversely related to the risk aversion
coefficient.  Higher reward (per unit of risk) convinces the consumer
to take the risk necessary to earn higher returns; but higher risk
aversion convinces the investor to sacrifice return for safety.


\appendix 

\begin{figure}[h]
\caption{The Risky Portfolio Share $\riskyshare$ and Relative Risk Aversion $\CRRA$} \label{fig:Port}\centering
\subfigure[The Risky Portfolio Share $\riskyshare$ Declines as Relative Risk Aversion $\CRRA$ Increases]{
    \label{fig:Port:a}
    \fbox{\CDCFig{ShareVsCRRA}}
}\\
\vspace{.1in} \subfigure[The Approximation Error for the Portfolio Share Is Small] {
    \label{fig:Port:b}
    \fbox{\CDCFig{ShareApproxErr}}
} \begin{flushleft} \footnotesize Note: The approximation error is computed by solving for the exactly optimal
portfolio share numerically.  See the \texttt{Portfolio-CRRA-Derivations.nb} Mathematica notebook for details.
\end{flushleft}
\end{figure}

\end{verbatimwrite}
\input ./body.tex


\input bibMake

\end{document}


